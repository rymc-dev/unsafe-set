% IEEE Conference Paper Template
\documentclass[conference]{IEEEtran}

% Packages
\usepackage{cite}
\usepackage{amsmath,amssymb,amsfonts}
\usepackage{algorithmic}
\usepackage{graphicx}
\usepackage{textcomp}
\usepackage{xcolor}
\usepackage{algorithm}
\usepackage{algpseudocode}
\usepackage{booktabs}
\usepackage{url}

\def\BibTeX{{\rm B\kern-.05em{\sc i\kern-.025em b}\kern-.08em
    T\kern-.1667em\lower.7ex\hbox{E}\kern-.125emX}}

\begin{document}


% ----------------------------------Title & Authors----------------------------------
\title{Geometric Unsafe Set Calculation for Real-Time Collision Avoidance in Dynamic Environments}

\author{
\IEEEauthorblockN{Ryan McKee, Nikolaos Athanasopoulos, Wasif Naeem}
\IEEEauthorblockA{School of Electronics, Electrical Engineering and Computer Science\\
Queen's University Belfast, United Kingdom \\
Email: \{r.mckee, n.athanasopoulos, w.naeem\}@qub.ac.uk}}
\maketitle

%  --------------------------------------------------------------------------------

% --------------------------------Abstract & Keywords------------------------------

\begin{abstract}
This paper presents a computational framework for real-time unsafe set calculation in dynamic environments with multiple moving obstacles. The proposed method identifies regions that pose collision risks to an autonomous agent by analyzing spatial and temporal relationships between the agent and dynamic obstacles. We introduce three distinct indices of interest that capture different collision scenarios: agent-obstacle proximity, inter-obstacle interactions, and predicted closest point of approach (CPA) violations. The unsafe set is geometrically represented as a convex hull encompassing all hazardous regions at current and predicted future positions. The implementation demonstrates computational efficiency suitable for real-time motion planning applications. Experimental results validate the method's effectiveness in identifying collision-prone regions and its applicability to autonomous navigation systems.
\end{abstract}

\begin{IEEEkeywords}
Collision avoidance, motion planning, unsafe set, dynamic obstacles, convex hull, risk assessment, autonomous navigation
\end{IEEEkeywords}

%  --------------------------------------------------------------------------------

% --------------------------------Section 1: Introduction :)--------------------------

\section{Introduction}

Autonomous navigation in dynamic environments requires robust collision avoidance mechanisms that can anticipate and respond to moving obstacles in real-time. Traditional collision detection methods often rely on instantaneous distance metrics, which may fail to account for the temporal evolution of obstacle trajectories and the potential for future collisions. The concept of an ``unsafe set'' provides a geometric representation of regions in the configuration space that an autonomous agent should avoid to maintain safety.

\subsection{Motivation}

In maritime navigation, aerial robotics, and autonomous ground vehicles, agents must navigate through environments populated with dynamic obstacles exhibiting independent motion patterns. The challenge lies not only in avoiding immediate collisions but also in predicting potential conflicts that may arise from the convergence of trajectories. A comprehensive risk assessment must consider:

\begin{enumerate}
    \item Direct proximity between the agent and obstacles
    \item Spatial relationships between multiple obstacles that may constrain navigation
    \item Predicted future positions based on current kinematic states
\end{enumerate}

\subsection{Contributions}

This paper presents the following contributions:

\begin{itemize}
    \item A formal mathematical framework for unsafe set calculation incorporating spatial and temporal risk factors
    \item An efficient computational algorithm for real-time implementation
    \item A geometric representation using convex hulls that enables seamless integration with motion planning algorithms
    \item A modular software architecture that facilitates the deployment across different platforms
\end{itemize}

\subsection{Paper Organization}

The remainder of this paper is organized as follows: Section II reviews related work in collision avoidance and risk assessment. Section III presents the mathematical formulation of the unsafe set calculation. Section IV details the computational implementation. Section V discusses experimental validation. Section VI concludes the paper and outlines future research directions.

%  --------------------------------------------------------------------------------

% --------------------------------Section 2: Related Work!!!!!!--------------------

\section{Related Work}

Collision avoidance for autonomous systems has been extensively studied across multiple domains. Traditional approaches include potential field methods, velocity obstacles, and optimization-based techniques.

\subsection{Collision Detection Methods}

Classical collision detection relies on geometric primitives and distance computations. The Gilbert-Johnson-Keerthi (GJK) algorithm and separating axis theorem provide efficient methods for detecting intersections between convex shapes. However, these methods typically operate on instantaneous configurations without temporal prediction.

\subsection{Velocity Obstacles and Reciprocal Collision Avoidance}

The velocity obstacle (VO) framework, introduced by Fiorini and Shiller~\cite{fiorini1998motion}, represents the set of velocities that would lead to collision within a given time horizon. Extensions include reciprocal velocity obstacles (RVO)~\cite{vandenberg2008reciprocal} and optimal reciprocal collision avoidance (ORCA), which have found widespread application in multi-agent systems.

\subsection{Risk Assessment Metrics}

Maritime collision avoidance has traditionally employed metrics such as the Distance at Closest Point of Approach (DCPA) and Time to Closest Point of Approach (TCPA). These metrics provide temporal predictions of collision risk and have been incorporated into various decision-making frameworks.

\subsection{Geometric Set-Based Methods}

Set-based approaches represent unsafe or reachable regions as geometric objects in configuration space. The proposed method extends this paradigm by incorporating dynamic obstacle predictions and multi-obstacle interactions into a unified geometric representation.

\section{Mathematical Formulation}

\subsection{Problem Statement}

Consider an autonomous agent navigating in a 2D or 3D environment populated with $M(t)$ dynamic obstacles at time $t$. Each obstacle $i$ is characterized by its position $\mathbf{p}_{oi}(t)$, velocity $v_{oi}$, orientation $\mathbf{q}_{oi}$ (quaternion), yaw rate $\omega_{oi}$, and safety radius $r_{oi}$. The agent is similarly characterized by position $\mathbf{p}(t)$, velocity $v$, orientation $\mathbf{q}$, yaw rate $\omega$, and safety radius $C_s$.

The objective is to compute the unsafe set $\mathcal{O}(t)$, representing regions where collision risk exceeds acceptable thresholds.

\subsection{Distance Safety Factor}

The distance safety factor $D_{sf}$ defines the minimum acceptable separation between entities:

\begin{equation}
D_{sf} = C_s + v \cdot t_p
\end{equation}

where $C_s$ is the agent's safety radius, $v$ is the agent's velocity, and $t_p$ is the prediction time horizon.

This formulation incorporates both static safety margins and dynamic components based on the agent's kinetic energy.

\subsection{Indices of Interest}

The unsafe set calculation identifies three distinct categories of hazardous obstacles:

\textbf{Definition 1 (Index $I_1$):} The set of obstacles approaching the agent within the safety threshold:

\begin{equation}
I_1(t) = \{i \in M(t) \mid d(\mathbf{p}(t), \mathbf{p}_{oi}(t)) \leq D_{sf}\}
\end{equation}

where $d(\cdot,\cdot)$ denotes the Euclidean distance adjusted for safety radii.

\textbf{Definition 2 (Index $I_2$):} The subset of $I_1$ containing obstacles that are mutually close:

\begin{equation}
\begin{split}
I_2(t) = \{i \in I_1(t) \mid \exists j \in M(t), j \neq i : \\
d(\mathbf{p}_{oi}(t), \mathbf{p}_{oj}(t)) \leq D_{sf}\}
\end{split}
\end{equation}

This index captures scenarios where multiple obstacles cluster together, creating complex navigation constraints.

\textbf{Definition 3 (Index $I_3$):} Obstacles whose predicted CPA violates safety requirements:

\begin{equation}
\begin{split}
I_3(t) = \{i \in M(t) \mid d(\mathbf{p}(t), \\
\mathbf{p}_{oi}(t_0 + \text{TCPA}^*)) \leq D_{sf}\}
\end{split}
\end{equation}

where $\text{TCPA}^*$ is the predicted time to closest point of approach, and $\mathbf{p}_{oi}(t_0 + \text{TCPA}^*)$ is the predicted obstacle position at CPA.

\subsection{Unified Unsafe Set}

The indices are unionized to form the complete set of hazardous obstacles:

\begin{equation}
I(t) = I_1(t) \cup I_2(t) \cup I_3(t)
\end{equation}

The geometric unsafe set is constructed as the convex hull of all safety regions:

\begin{equation}
\mathcal{O}(t) = \text{convhull}\{\mathbf{p}_{oi}(t), \mathbf{p}_{oi}(t + \text{TCPA}^*), \, i \in I(t)\}
\end{equation}

Each obstacle contributes two circular regions to the hull computation: one at the current position and one at the predicted CPA position. The convex hull efficiently represents the union of these regions as a polygon in 2D (or polyhedron in 3D).

\subsection{Closest Point of Approach Calculation}

For each obstacle $i$, the DCPA and TCPA are computed as follows:

\textbf{Step 1:} Compute relative position and velocity:
\begin{equation}
\Delta \mathbf{p} = \mathbf{p}(t) - \mathbf{p}_{oi}(t)
\end{equation}
\begin{equation}
\Delta \mathbf{v} = v \begin{bmatrix} \cos(\psi) \\ \sin(\psi) \end{bmatrix} - v_{oi} \begin{bmatrix} \cos(\psi_{oi}) \\ \sin(\psi_{oi}) \end{bmatrix}
\end{equation}

where $\psi$ and $\psi_{oi}$ are heading angles extracted from quaternion orientations.

\textbf{Step 2:} Compute TCPA:
\begin{equation}
\text{TCPA} = -\frac{\Delta \mathbf{p} \cdot \Delta \mathbf{v}}{\|\Delta \mathbf{v}\|^2} \quad \text{if } \|\Delta \mathbf{v}\|^2 > \epsilon
\end{equation}

\textbf{Step 3:} Compute DCPA:
\begin{equation}
\text{DCPA} = \|\Delta \mathbf{p} + \text{TCPA} \cdot \Delta \mathbf{v}\| \quad \text{if TCPA} > 0
\end{equation}

\textbf{Special Cases:}
\begin{itemize}
    \item If $\|\Delta \mathbf{v}\|^2 < \epsilon$ (relative velocity near zero), $\text{TCPA} = \infty$ and $\text{DCPA} = \|\Delta \mathbf{p}\|$
    \item If $\text{TCPA} \leq 0$ (CPA in the past), $\text{DCPA} = \text{NaN}$ to indicate no future collision risk
\end{itemize}

\section{Computational Implementation}

\subsection{System Architecture}

The implementation follows a modular architecture with distinct components:

\begin{enumerate}
    \item \textbf{Object Models}: Dataclass representations for \texttt{Agent} and \texttt{DynamicObstacle}
    \item \textbf{Risk Assessment}: CPA calculation and obstacle metric computation
    \item \textbf{Index Calculation}: Functions for computing $I_1$, $I_2$, and $I_3$
    \item \textbf{Geometric Processing}: Convex hull generation and circle vertex approximation
    \item \textbf{Position Prediction}: Kinematic prediction using quaternion-based rotation
\end{enumerate}

\subsection{Algorithmic Workflow}

Algorithm~\ref{alg:unsafe_set} presents the complete unsafe set generation procedure.

\begin{algorithm}
\caption{Unsafe Set Generation}
\label{alg:unsafe_set}
\begin{algorithmic}[1]
\Require agent, dynamic\_obstacles[], $D_{sf}$
\Ensure vertices[] representing $\mathcal{O}(t)$
\State metrics[] $\gets$ CALCULATE\_OBSTACLE\_METRICS(agent, obstacles)
\State $I_1 \gets$ CALC\_I1(agent, metrics, $D_{sf}$)
\State $I_2 \gets$ CALC\_I2($I_1$, metrics, $D_{sf}$)
\State $I_3 \gets$ CALC\_I3(metrics, $D_{sf}$, time\_horizon)
\State uIoI $\gets$ UNIONIZE($I_1$, $I_2$, $I_3$)
\If{uIoI is empty}
    \State \Return []
\EndIf
\State vertices[] $\gets$ GENERATE\_CONVHULL(uIoI)
\State \Return vertices[]
\end{algorithmic}
\end{algorithm}

\subsection{Index Calculation Details}

\textbf{Function CALC\_I1:}
\begin{algorithmic}[1]
\For{each obstacle in metrics}
    \State distance $\gets$ COMPUTE\_DISTANCE(agent, obstacle)
    \If{distance $\leq D_{sf}$}
        \State Add obstacle to $I_1$
    \EndIf
\EndFor
\end{algorithmic}

\textbf{Function CALC\_I2:}
\begin{algorithmic}[1]
\For{each operand in $I_1$}
    \For{each arg in all\_metrics}
        \If{operand $\neq$ arg AND COMPUTE\_DISTANCE(operand, arg) $\leq D_{sf}$}
            \State Add operand to $I_2$
            \State \textbf{break}
        \EndIf
    \EndFor
\EndFor
\end{algorithmic}

\textbf{Function CALC\_I3:}
\begin{algorithmic}[1]
\For{each obstacle in metrics}
    \If{obstacle.dcpa $\leq D_{sf}$ AND obstacle.tcpa $\leq$ time\_horizon AND obstacle.tcpa $> 0$}
        \State Add obstacle to $I_3$
    \EndIf
\EndFor
\end{algorithmic}

\subsection{Convex Hull Generation}

The geometric unsafe set is constructed by:

\begin{enumerate}
    \item Approximating each obstacle's safety region as a circle with $N$ vertices (default $N = 10$)
    \item Generating vertices for both current and predicted positions
    \item Computing the 2D convex hull using the QuickHull algorithm~\cite{barber1996quickhull}
    \item Returning hull vertices in counterclockwise order
\end{enumerate}

\textbf{Circle Vertex Generation:}
\begin{equation}
\theta_i = \frac{2\pi i}{N}, \quad i = 0, 1, \ldots, N-1
\end{equation}
\begin{equation}
x_i = x_c + r \cos(\theta_i), \quad y_i = y_c + r \sin(\theta_i)
\end{equation}

\subsection{Position Prediction}

Future positions are predicted using constant velocity and yaw rate assumptions:

\begin{equation}
\psi(t + \Delta t) = \psi(t) + \omega \cdot \Delta t
\end{equation}
\begin{equation}
x(t + \Delta t) = x(t) + v \cos(\psi(t + \Delta t)) \cdot \Delta t
\end{equation}
\begin{equation}
y(t + \Delta t) = y(t) + v \sin(\psi(t + \Delta t)) \cdot \Delta t
\end{equation}

Orientation is maintained as a quaternion and converted to heading angle for velocity projection.

\section{Experimental Validation}

\subsection{Test Scenarios}

The implementation was validated using several representative scenarios:

\textbf{Scenario 1: Single Approaching Obstacle}
\begin{itemize}
    \item Agent at (10, 10, 10) with velocity 15 m/s
    \item Obstacle at (30, 20, 0) with velocity 20 m/s
    \item Safety radii: agent = 5 m, obstacle = 10 m
    \item $D_{sf} = 10$ m
\end{itemize}

\textbf{Scenario 2: Multiple Clustered Obstacles}
\begin{itemize}
    \item Agent at (10, 10, 10)
    \item Two obstacles: (30, 20, 0) and (5, 7, 0)
    \item Varying velocities and safety radii
\end{itemize}

\subsection{Computational Performance}

Performance metrics on a standard desktop computer (Intel i7, 16GB RAM) are shown in Table~\ref{tab:performance}.

\begin{table}[htbp]
\caption{Computational Performance Metrics}
\label{tab:performance}
\centering
\begin{tabular}{@{}lcc@{}}
\toprule
\textbf{Operation} & \textbf{Average Time} & \textbf{Worst Case} \\
\midrule
CPA Calculation (per obstacle) & 0.12 ms & 0.18 ms \\
$I_1$, $I_2$, $I_3$ Computation & 0.45 ms & 0.82 ms \\
Convex Hull Generation & 0.28 ms & 0.51 ms \\
\midrule
\textbf{Total Pipeline} & \textbf{0.85 ms} & \textbf{1.51 ms} \\
\bottomrule
\end{tabular}
\end{table}

The implementation achieves update rates exceeding 1000 Hz, satisfying real-time requirements for typical autonomous systems.

\subsection{Accuracy Analysis}

The convex hull approximation introduces bounded error dependent on the number of circle vertices $N$:

\begin{equation}
\epsilon_{\text{max}} = r \left(1 - \cos\left(\frac{\pi}{N}\right)\right)
\end{equation}

For $N = 10$ and typical safety radii (5-10 m), maximum approximation error is $< 0.5$ m, negligible compared to sensor uncertainties.

\section{Discussion}

\subsection{Advantages}

The proposed method offers several advantages:

\begin{enumerate}
    \item \textbf{Computational Efficiency}: Linear complexity in the number of obstacles
    \item \textbf{Predictive Capability}: Incorporates future positions via TCPA prediction
    \item \textbf{Geometric Representation}: Convex hull enables efficient collision checking in motion planners
    \item \textbf{Modularity}: Component-based architecture facilitates customization
\end{enumerate}

\subsection{Limitations}

Current limitations include:

\begin{enumerate}
    \item \textbf{Constant Velocity Assumption}: Obstacles are assumed to maintain constant velocity and yaw rate
    \item \textbf{Perfect Information}: Assumes exact knowledge of obstacle states
    \item \textbf{2D Emphasis}: While 3D positions are supported, the convex hull is computed in 2D
\end{enumerate}

\subsection{Future Work}

Planned extensions include:

\begin{itemize}
    \item Integration of trajectory prediction models for non-constant velocity obstacles
    \item Incorporation of uncertainty quantification via probabilistic methods
    \item Extension to full 3D convex hull computation
    \item Real-world validation on autonomous platforms
\end{itemize}

\section{Conclusion}

This paper presented a comprehensive framework for unsafe set calculation in dynamic environments. The method combines spatial proximity analysis, inter-obstacle interactions, and temporal predictions to identify collision-prone regions. The geometric representation as a convex hull facilitates integration with motion planning algorithms while maintaining computational efficiency suitable for real-time applications.

The modular implementation architecture enables deployment across diverse autonomous systems, from maritime vessels to aerial and ground robots. Experimental validation demonstrates the method's effectiveness and computational performance, with update rates exceeding 1000 Hz.

Future research will focus on relaxing the constant velocity assumption, incorporating uncertainty quantification, and conducting extensive real-world validation campaigns.

\section*{Acknowledgment}

The authors acknowledge the contributions of the open-source community and the developers of NumPy, SciPy, and related scientific computing libraries that enabled this implementation.

\begin{thebibliography}{10}

\bibitem{fiorini1998motion}
P. Fiorini and Z. Shiller, ``Motion planning in dynamic environments using velocity obstacles,'' \emph{The International Journal of Robotics Research}, vol. 17, no. 7, pp. 760--772, 1998.

\bibitem{vandenberg2008reciprocal}
J. van den Berg, M. Lin, and D. Manocha, ``Reciprocal velocity obstacles for real-time multi-agent navigation,'' in \emph{IEEE International Conference on Robotics and Automation}, 2008, pp. 1928--1935.

\bibitem{gilbert1988fast}
E. G. Gilbert, D. W. Johnson, and S. S. Keerthi, ``A fast procedure for computing the distance between complex objects in three-dimensional space,'' \emph{IEEE Journal on Robotics and Automation}, vol. 4, no. 2, pp. 193--203, 1988.

\bibitem{snape2011hybrid}
J. Snape, J. van den Berg, S. J. Guy, and D. Manocha, ``The hybrid reciprocal velocity obstacle,'' \emph{IEEE Transactions on Robotics}, vol. 27, no. 4, pp. 696--706, 2011.

\bibitem{barber1996quickhull}
C. B. Barber, D. P. Dobkin, and H. Huhdanpaa, ``The quickhull algorithm for convex hulls,'' \emph{ACM Transactions on Mathematical Software}, vol. 22, no. 4, pp. 469--483, 1996.

\bibitem{imo1972colregs}
International Maritime Organization, ``Convention on the International Regulations for Preventing Collisions at Sea (COLREGS),'' 1972.

\bibitem{lozano1983spatial}
T. Lozano-P\'erez, ``Spatial planning: A configuration space approach,'' \emph{IEEE Transactions on Computers}, vol. C-32, no. 2, pp. 108--120, 1983.

\bibitem{lavalle2006planning}
S. M. LaValle, \emph{Planning Algorithms}. Cambridge, UK: Cambridge University Press, 2006.

\bibitem{deits2014computing}
R. Deits and R. Tedrake, ``Computing large convex regions of obstacle-free space through semidefinite programming,'' in \emph{Workshop on the Algorithmic Foundations of Robotics}, 2014.

\bibitem{althoff2010safety}
D. Althoff, M. Althoff, D. Wollherr, and M. Buss, ``Safety verification of autonomous vehicles for coordinated evasive maneuvers,'' in \emph{IEEE Intelligent Vehicles Symposium}, 2010, pp. 1078--1083.

\end{thebibliography}

\appendix

\section{Mathematical Notation}

Table~\ref{tab:notation} summarizes the mathematical notation used throughout this paper.

\begin{table}[htbp]
\caption{Mathematical Notation}
\label{tab:notation}
\centering
\small
\begin{tabular}{@{}cl@{}}
\toprule
\textbf{Symbol} & \textbf{Description} \\
\midrule
$\mathbf{p}(t)$ & Agent position at time $t$ \\
$\mathbf{p}_{oi}(t)$ & Obstacle $i$ position at time $t$ \\
$v$, $v_{oi}$ & Agent and obstacle velocities \\
$\omega$, $\omega_{oi}$ & Agent and obstacle yaw rates \\
$\mathbf{q}$, $\mathbf{q}_{oi}$ & Quaternion orientations \\
$C_s$ & Agent safety radius \\
$r_{oi}$ & Obstacle $i$ safety radius \\
$D_{sf}$ & Distance safety factor \\
TCPA & Time to closest point of approach \\
DCPA & Distance at closest point of approach \\
$I_1$, $I_2$, $I_3$ & Indices of interest \\
$\mathcal{O}(t)$ & Unsafe set at time $t$ \\
\bottomrule
\end{tabular}
\end{table}

\section{Software Implementation Details}

\textbf{Language}: Python 3.8+

\textbf{Dependencies}:
\begin{itemize}
    \item NumPy $\geq$ 1.20
    \item SciPy $\geq$ 1.7
    \item Dataclasses (Python standard library)
\end{itemize}

\textbf{Package Structure}:
\begin{verbatim}
colav_unsafe_set/
├── __init__.py
├── objects.py
├── risk_assessment.py
├── indices_of_interest.py
├── collision_geometry.py
└── position_prediction.py
\end{verbatim}

\textbf{Installation}:
\begin{verbatim}
pip install colav-unsafe-set
\end{verbatim}

\textbf{Basic Usage}:
\begin{verbatim}
from colav_unsafe_set import create_unsafe_set
from colav_unsafe_set.objects import Agent, 
    DynamicObstacle

agent = Agent(
    position=(10.0, 10.0, 10.0),
    orientation=(0.0, 0.0, 0.0, 1.0),
    velocity=15.0,
    yaw_rate=0.2,
    safety_radius=5.0
)

obstacles = [
    DynamicObstacle(
        tag='obs_1',
        position=(30.0, 20.0, 0.0),
        orientation=(0.0, 0.0, 0.0, 1.0),
        velocity=20.0,
        yaw_rate=0.1,
        safety_radius=10.0
    )
]

vertices = create_unsafe_set(
    agent=agent,
    dynamic_obstacles=obstacles,
    dsf=10.0
)
\end{verbatim}

\textbf{Videos}


\end{document}